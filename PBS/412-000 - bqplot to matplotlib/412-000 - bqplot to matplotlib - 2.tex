\documentclass[10pt,a4paper]{article}
\usepackage[utf8]{inputenc}
\usepackage[basque]{babel}
\usepackage{amsmath}
\usepackage{amsfonts}
\usepackage{amssymb}
\usepackage{makeidx}
\usepackage{graphicx}
\graphicspath{ {images/} }
\usepackage{lmodern}
\usepackage[left=3cm,right=3cm,top=3cm,bottom=2cm]{geometry}
\usepackage{fancyhdr}
\usepackage{montserrat}
\usepackage{lastpage}
\usepackage{afterpage}
\newcommand\blankpage{%
	\null
	\thispagestyle{empty}%
    \addtocounter{page}{-1}%
	\newpage}

\usepackage[export]{adjustbox}
\usepackage{wrapfig}
\usepackage{hyperref}

\usepackage{nameref}

\setlength\parindent{0pt}

\usepackage{xcolor}
\usepackage{titlesec}
  
\let\nf\normalfont %\nf komandoari \nf deitu  
\newcommand{\cf}{\normalfont\sffamily}

  
\titleformat{\section}
  {\nf\sffamily\Large}
  {\thesection}{1em}{}

\titleformat{\subsection}
  {\nf\sffamily\Large}
  {\thesubsection}{1em}{}

\renewcommand{\footrulewidth}{0pt}

\pagestyle{fancy}
\fancyhf{}
\lhead{\cf 2019/10/15}
\chead{\cf v 1.0}
\rhead{\cf \# 412-000}
\fancyfoot[C]{\cf\thepage /\pageref{LastPage}}


\title{\cf \# 412-000 \\ \vspace{5mm}
					Exportatzeko modulua: bqplot to matplotlib\\
\normalsize \vspace{5mm} Data: 2019/10/15 \\
 			\vspace{3mm} Bertsioa: 1.0}
\date{}
%\author{Jon Gabirondo López}

\begin{document}

\maketitle
\thispagestyle{fancy}
%\author

\section{Helburua}
Modulo honen helburua bqplot-eko Figure bat jaso eta matplotlib erabiliz aipaturiko figuraren kalitate handiko irudi bat sortzea da. Figura jasotzean bere barnean dituen marken (lerroak, scatter-ak etab.) itxura aldatzea ahalbidetuko du eta egindako aldaketak bqplot-eko figura batean zein amaierako irudiaren berdina izango den matplotlib-en irudi batean ikusi ahalko dira.
\\

Modulo hau MinervaLab-eko programa askorekin zein CallistoLab-ekin konpatiblea izateko diseinatu da.

\section{Garapena}

\begin{itemize}

\item \cf \#412-001: Izenburua aldatzeko textu-formularioa.
\\
\nf Formulario honek une horretan figurak duen izenburua erakutsiko du eta aldatzean \cf \# 412-004 \nf elementuan zuzenean aldatuko da.

\item \cf \#412-002: Ardatzen izenak aldatzeko textu-formularioa.
\\
\nf Formulario honek une horretan figurak duen izenburua erakutsiko du eta aldatzean \cf \# 412-004 \nf elementuan zuzenean aldatuko da.

\item \cf \#412-003: Ardatzetako tick-en formatoa aldatzeko checkbox-ak.
\\
\nf Sakatzean ardatzetako tick-en balioak borobilduta agertuko dira \cf \# 412-004 \nf elementuan.

\textit{Agian tick-en hamartar kopurua zehazteko dropdown-ak jarri daitezke.}

\item \cf \#412-004: bqplot-eko figura.
\\
\nf Figuraren uneoroko egoera erakusten duen bqplot-eko figura. Matplotlib-ekin interakzioa askoz zailagoa denez, aldaketen previsualizazioa elementu honen bidez eskaintzen da eta, gero, nahi denean \cf \# 412-005 \nf elementura pasatzen dira \cf \# 412-00B \nf botoia sakatzean.

\item \cf \#412-005: matplotlib-eko figura.
\\
\nf Lortuko den irudiaren egoera erakusten duen matplotlib-eko figura. \cf \# 412-00B \nf botoia sakatzean \cf \#412-004 \nf -ko marka guztien egoera honera bidaltzen dira.

\item \cf \#412-006: Markaren izena aldatzeko textu-formularioa.
\\
\nf Formulario honek une horretan markak duen izena erakutsiko du eta aldatzean leyendako balioa aldatuko da.

\item \cf \#412-007: Datuen estiloa aldatzeko dropdown-a.
\\
\nf Markaren uneoroko estiloa erakusten du (edo errepresentazio mota). Aukerak 'Scatter' eta 'Line' izango dira.
\\

\item \cf \#412-008: Markaren zabalera aldatzeko slider-a.
\\
\nf Hau aldatzean markaren zabalera aldatuko da \cf \# 412-004 \nf elementuan.
\\

\item \cf \#412-009: Markaren opazitatea aldatzeko slider-a.
\\
\nf Hau aldatzean markaren opazitatea aldatuko da \cf \# 412-004 \nf elementuan.

\item \cf \#412-00A: Markaren kolorea aldatzeko formularioa (color-picker-a).
\\
\nf Hau aldatzean markaren kolorea aldatuko da \cf \# 412-004 \nf elementuan.

\item \cf \#412-00B: matplotlib-eko figura eguneratzeko botoia.
\\
\nf Botoia sakatzean \cf \# 412-004 \nf elementuko marka guztiak eta haien egoera \cf \# 412-005 \nf elementura pasako dira.
\\

\item \cf \#412-00C: Sortuko den fitxategiaren izena idazteko textu-formularioa.
\\
\nf \cf \# 412-004 \nf -ko botoiak sakatzean sortzen diren fitxategien izena bezala formulario honen balioa erabiliko da (formatuaren atzizki egokiarekin, noski)
\\

\item \cf \#412-00D: Fitxategiak sortzeko botoiak.
\\
\nf Sakatzean dagokion formatuko fitxategi bat sortu eta hura deskargatzeko beste botoi bat agertuko da.
\end{itemize}
\end{document}