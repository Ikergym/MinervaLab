\documentclass[10pt,a4paper]{article}
\usepackage[utf8]{inputenc}
\usepackage[basque]{babel}
\usepackage{amsmath}
\usepackage{amsfonts}
\usepackage{amssymb}
\usepackage{makeidx}
\usepackage{graphicx}
\graphicspath{ {images/} }
\usepackage{lmodern}
\usepackage[left=3cm,right=3cm,top=3cm,bottom=2cm]{geometry}
\usepackage{fancyhdr}
\usepackage{montserrat}
\usepackage{lastpage}
\usepackage{afterpage}
\newcommand\blankpage{%
	\null
	\thispagestyle{empty}%
    \addtocounter{page}{-1}%
	\newpage}

\usepackage[export]{adjustbox}
\usepackage{wrapfig}
\usepackage{hyperref}

\usepackage{nameref}

\usepackage{tabularx} %multilie tables egiteko
    \newcolumntype{L}{>{\raggedright\arraybackslash}X}

\setlength\parindent{0pt}

\usepackage{xcolor}
\usepackage{titlesec}
\usepackage{float}
 
\let\nf\normalfont %\nf komandoari \nf deitu  
\newcommand{\cf}{\normalfont\sffamily}

  
\titleformat{\section}
  {\nf\sffamily\Large}
  {\thesection}{1em}{}

\titleformat{\subsection}
  {\nf\sffamily\Large}
  {\thesubsection}{1em}{}

\renewcommand{\footrulewidth}{0pt}

\pagestyle{fancy}
\fancyhf{}
\lhead{\cf 2019/11/06}
\chead{\cf v 2.1}
\rhead{\cf \# 111-000}
\fancyfoot[C]{\cf\thepage /\pageref{LastPage}}


\title{\cf \# 111-000 \\ \vspace{5mm}
					Van der Waals-en egoera ekuazioaren azterketa 2 dimentsiotan: p(v, T)\\
\normalsize \vspace{5mm} Data: 2019/11/06 \\
 			\vspace{3mm} Bertsioa: 2.1}
\date{}
%\author{Jon Gabirondo López}

\begin{document}

\maketitle
\thispagestyle{fancy}
%\author

\begin{table}[H]
    \centering
\begin{tabularx}{\linewidth}{|c|c|L|} 
    \hline
\cf Bertsioa & \cf Data & \cf Deskribapena \\ \hline
1.0 & 2019/09/16 & \#111-000ren lehenengo bertsioa. \\  \hline
2.0 & 2019/11/04 & Berriak: \begin{itemize}
\item \#111-012
\item \#111-013
\item \#111-014
\end{itemize}
Aldatutakoak: \begin{itemize}
\item \#111-003: pausua aukeratzeko inputa gehitu
\item \#111-00A: tracer-aren posizioa erakusteko textua gehitu.
\end{itemize}
Kendutakoak: \begin{itemize}
\item \#111-001
\item \#111-004
\item \#111-007
\end{itemize} \\ \hline
2.1 & 2019/11/06 & Berriak: \begin{itemize}
\item \#111-015
\item \#111-016
\item \#111-017
\item \#111-018
\end{itemize}
Aldatutakoak: \begin{itemize}
\item \#111-003: isoterma kopurua aukeratzeko sliderra gehitu
\item \#111-004: berreskuratuta
\item \#111-009: 'Undo' botoia kendu
\end{itemize}
Kendutakoak: \begin{itemize}
\item \#111-002
\item \#111-012
\item \#111-013: Koordenatu absolutuena beste programa batean joango da. Ikusi \#111-002-ean idatzitako iruzkina.
\end{itemize} \\ \hline
\end{tabularx}
    \end{table}

\newpage

\maketitle
\thispagestyle{fancy}

\section{Helburua}
Programa honen helburua grafika interaktibo baten bitartez ikasleari Van der Waals-en egoera ekuazio mekanikoan agertzen diren parametro zein aldagaiek duten esangura fisikoataz jabetzea da. Aipaturiko aldagaiek sistemaren fisikan duten eragina momentuan ikusteak lehen ebazteko neketsuagoak ziren ariketak plantetzeko aukera ematen zaie irakasleei.
\\

\section{Oinarri teorikoa}
Honakoa da Van der Waals-en egoera ekuazio mekanikoa:

\begin{equation}\label{van_der_waals}
\left( p + \frac{a}{v^2} \right) ( v- b) = RT \qquad \text{non} \qquad R = \text{gas idealen konstantea}
\end{equation}

Sarritan, ekuazioaren forma laburtua erabiltzen da, batez ere esanguratsuak diren adieazpen grafikoak lortzeko.

\begin{equation}
p_R = \frac{p}{p_c} \qquad v_R = \frac{v}{v_c} \qquad T_R = \frac{T}{T_c}
\end{equation}

Aldagai erlatiboak definituz eta \ref{van_der_waals} ekuazioan ordezkatuz honako lortu dezakegu:

\begin{equation}\label{van_der_waals_R}
\left( p_R + \frac{3}{v_R^2} \right) ( v_R - \frac{1}{3}) = \frac{8}{3} T_R 
\end{equation}

non $p_c$, $v_c$ eta $T_c$ gasaren presio, volumen eta tenperatura-kritikoak diren, hurrenez-hurren:

\begin{equation}
p_C = \frac{1}{27} \frac{a}{b^2} \qquad v_c = 3b \qquad T_c = \frac{8}{27}\frac{a}{b}\frac{1}{R}
\end{equation}

Sarritan lortutako \ref{van_der_waals_R} ekuazioa erabiltzen da gas baten lerro isotermoak grafikatzeko. 

%Tenperatura kritikoaren azpitik zonalde batean gasaren egoerak ez du egonkortasun baldintza beteko.

ETAB. 
\section{Garapena}

\begin{itemize}

\item \cf \#111-002: a eta b parametroak eskuz aldatzeko sliderrak. (KENDUTA)
\\
\nf $a$ eta $b$ parametroak eskuz aldatzeko aukera egongo da. Bi slider hauek \cf \# 111-005 \nf grafikara lotuta egongo dira eta
momentuan ikusiko da parametro horiek aldatzearen eragina.

\textit{Hau beste programa batera pasako dugu, zeinak koordenatu absolutuetan isotermak erakutsiko dituen. Bertan erabiltzaileak parametro hauek duten eragina ikusi ahalko du eta behin $a$ eta $b$ finkatuta dituela \cf \#111-000 \nf ri bidali. Programa honetan ere \cf \#111-001 \nf egongo da (datu basetik a eta b aukeratzeko dropdown-a) eta slider-en mugak aukeratutako materialaren araberakoak izango dira (balio errealaren \%10, esaterako) lortutako grafikak egokiak izateko.}

\item \cf \#111-003: Tenperatura tartea aurkeratzeko textu-formularioak.
\\
\nf Kalkulatu nahi den tenperaturaren tartea, $T_c$-rekiko erlatiboki adierazita. Gainera, kalkulatuko diren isotermen kopurua aukeratzeko slider bat egongo da. Sliderraren tartea zein Tren tartea oso mugatuta egongo dira kalkulu kopurua murrizteko. 
\\

\item \cf \#111-005: p(v,T) planoaren grafika.
\\
\nf Grafika honetan une horretan lan egiten ari zaren lerro isotermoa, gordetako isoterma guztiak eta isoterma horietan gordetako markadoreak agertuko dira. Gainera, Maxwell-en 'construction'-eko lerro horizontala (isoterma erreala lortzeko erabiltzen dena) agertuko da eta zuzen hori eta lerro isotermaren arteko azalera koloreztatuta agertuko da.

\item \cf \#111-006: Lerro isoterma ezberdinekin lan egiteko slider-a.
\\
\nf Sliderrak une batean lan egiten ari garen lerro isotermoaren tenperatura aldatzeko balioko du.

\item \cf \#111-008: Kalkulatutako isoterma guztiak erakusteko checkbox-a.
\\
\nf Aktibatuta dagoenean \cf \#111-003 \nf elementuan aukeratutako tartean kalkulatutako isoterma guztiak erakutsiko dira.
\\

\textit{Zenbat isoterma kalkulatu behar diren aukeratzeko textu formulario bat jarri beharko litzateke.}

\item \cf \#111-009: Isotermak gordetzeko botoiak.
\\
\nf Lan egiten ari zaren isoterma gordetzeko botoia. Isoterma bat gordetzean \cf \#111-009 \nf an aukeratu bezala agertuko da.

\item \cf \#111-00A: Isoterman 'tracer' bat erakusteko checkbox-a eta hori mugitzeko slider-a.
\\
\nf Checkbox-a aktibatzean une horretan lan egiten ari garen isoterman puntu bat agertuko da, sliderraren bitartez isoterman zehar mugitu ahalko dena bere posizioa ondoko textuan ikusi ahal den bitartean.
\\

\textit{Tracer-a isoterma esperimental edo teorikotik mugitzeko aukera egiteko input bat jarri behar da.}

\item \cf \#111-00B: Marka bat gehitzeko formularioa.
\\
\nf Botoiari sakatzean textu formularioan jasotako izena duen puntu bat markatuko da 'tracer'-a dagoen tokian.
\\

\item \cf \#111-00C: 'Maxwell contruction' (palankaren erregela) erakusteko checkbox-a.
\\
\nf Aktibatzean palankaren erregelaren erlazionatutako elementuak ikusiko dira \cf \#111-005 \nf grafikan (presio konstanteko zuzena, koloreztatutako azalerak eta fase trantsizioaren mugak, azken hauek aktibatuta badaude).

\item \cf \#111-00D: presio konstanteko zuzena mugitzeko sliderra.
\\
\nf

\item \cf \#111-00E: Isotermaren eta presio konstanteko lerroaren arteko azalerak koloreztatzeko checkbox-a.
\\
\nf Aktibatzean presio konstantzeko lerroaren eta une horretan tenperaturaren sliderrean aukeratuta dagoen tenperaturaren arteko azalera koloreztatuko du.

\item \cf \#111-00F: Isotermaren eta presio konstanteko lerroaren arteko azaleren balioak adierazteko textuak.
\\
\nf Isotermaren eta presio konstanteko lerroaren arteko azaleren balioak agertuko dira: zuzenak aldapa konstanteko isotermaren zatia mozten duenetik ezkerrera gelditzen denaren azalera, bertatik eskubira gelditzen deneko azalera eta bien batura.
\\

Integral guztiak analitikoki ebatziko dira.
\\

\item \cf \#111-010: Isoterma erreala kalkulatzeko botoia (p finkatuta).
\\
\nf Sakatzean, programa arduratuko da tenperatura ezberdinetarako integralak burutzeaz eta integrala 0 deneko tenperatura topatzeaz. Agian, prozedura grafikan ikusi daiteke.
\\

Isoterma hau kalkulatzean kalkulatutako isotermei gehituko zaie eta haiek bezala jokatuko da (agian ezingo da isoterma honera Tren sliderraren bidez iritsi).


\item \cf \#111-011: Irudia exportatzeko botoia.
\\
\nf Sakatzean lortutako irudia formatu ezberdinetara exportatzea ahalbdeituko duen menu bat agertuko da.

\item \cf \#111-014: Isoterma erreala kalkulatzeko botoia (T finkatuta).
\\
\nf Sakatzean, programa arduratuko da presio ezberdinetarako integralak burutzeaz eta integrala 0 deneko presioa topatzeaz. Agian, prozedura grafikan ikusi daiteke.

\item \cf \#111-015: Isoterma teorikoak edota esperimentalak erakusteko botoiak.
\\
\nf Toggle button-ak erabiliko ditugu: bitako bat edo biak sakatuta egon ahalko dira. Defektuz, isoterma teorikoak erakutsiko dira. Isoterma guztiak hasieran kalkulatuko dira, \cf \#111-004 \nf botoia sakatzean.

\item \cf \#111-016: Ikusgarri dauden isotermak aukeratzeko 'aukeratzailea' (selector).
\\
\nf Multiple selector bat erabiliko dugu: control/shift erabiliz hainbat isoterma aukeratu ahalko ditugu eta aukeratutakoak bakarrik ikusiko dira grafikan.

\item \cf \#111-017: Fase transizioko zonaldearen mugak erakusteko checkbox-a.
\\
\nf Hau sakatzean (eta \cf \#111-00C \nf) aktibatuta badago, fase trantsizioaren mugak adierazteko bi puntu agertuko dira isotermaren tenperatura tenperatura kritikotik behera duten eta ikusgarri dauden isoterma guztietan.

\item \cf \#111-018: Sortutako marketatik ikusgarri daudenak aukeratzeko 'aukeratzailea'.
\\
\nf Multiple selector bat erabiliko dugu: control/shift erabiliz sortutako marketatik hainbat aukeratu ahalko ditugu eta aukeratutakoak bakarrik ikusiko dira grafikan.
\end{itemize}
\end{document}